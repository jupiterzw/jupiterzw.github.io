\documentclass[11pt]{article}
% \usepackage{pm_exam_24}

\begin{document}
% \begin{titlepage}
%  \begin{tikzpicture}[remember picture, overlay]
%     \node[opacity=0.3, inner sep=0pt] at (current page.center){
%         \includegraphics[width=\paperwidth,height=\paperheight]{IMG_4798.PNG}
%     };
%  \end{tikzpicture}
%  \centering

%  % Title with large font size and rules
%  \rule{\textwidth}{1.6pt}\\[-3mm]
%  \rule{\textwidth}{0.4pt}\\[5mm]
% {\Huge \textbf{ \fontsize{12mm}{12mm} MATH0029 Graph Theory Mock Exam}}\\[5mm]
%  \rule{\textwidth}{0.4pt}\\[-3mm]
%  \rule{\textwidth}{1.6pt}\\[2cm]

%  % Course, Lecturer, and Scribe Information
%  \vfill

%  % Bottom Information
%  \textbf{To AX $\heartsuit$}
% \end{titlepage}

% Contents
\newpage
\pagenumbering{gobble}
\section*{Question 1}

\begin{enumerate}
    \item[(a)] (\textbf{3 marks}) Let $G$ be a graph and $k \geq 1$ an integer. What does it mean to say that (i) $G$ is \textit{$k$-colourable}; (ii) $G$ has \textit{chromatic number} $\chi(G) = k$.
    \item[(b)] (\textbf{4 marks}) Give an example of a triangle-free graph $G$ with $\chi(G) = 3$. 
    \item[(c)] (\textbf{7 marks}) Prove that if $G$ has maximum degree $\Delta(G)$, then $\chi(G) \leq \Delta(G) + 1$.
    \item[(d)] (\textbf{11 marks}) Prove that if $G$ is a graph with $m$ edges, then $G$ contains a $3$-colourable subgraph with at least $\lceil 2m / 3\rceil$ edges.
\end{enumerate}

\section*{Question 2}

\begin{enumerate}
    \item[(a)] (\textbf{2 marks}) Define $\text{ex}(n, H)$, where $H$ is a graph and $n \geqslant 1$ is an integer.
    \item[(b)] (\textbf{5 marks}) Define the Turán graph $T_r(n)$. How many edges are there in $T_4(11)$?
    \item[(c)] (\textbf{8 marks}) Given a graph $H$ define $\pi(H)$, the Turán density of $H$, and prove that $\pi(H)$ is well-defined.
    \item[(d)] (\textbf{10 marks}) Show that if $G$ is a $K_3$-free graph of order $2n$ with $ n^2 -t$ edges, for some $t \geqslant 0$, then $G$ contains a bipartite subgraph with at least $n^2-2 t$ edges.
\end{enumerate}

\section*{Question 3}

\begin{enumerate}
    \item[(a)] (\textbf{3 marks}) What does it mean to say that a family of sets is (i) an \textit{antichain}; (ii) a \textit{chain}.
    \item[(b)] (\textbf{5 marks}) Prove that if $n \geqslant 1$ is an integer and $\mathcal{C}$ is an chain in $\mathcal{P}([n])$, then
    \[
    |\mathcal{C}| \leqslant
    n+1
    .
    \]
    \item[(c)] (\textbf{7 marks}) Give a symmetric chain decomposition of $\mathcal{P}([4])$.
    \item[(d)] (\textbf{3 marks}) State Sperner's Theorem.
    \item[(e)] (\textbf{7 marks}) Let $X=\left\{x_1, x_2, x_3, x_4, x_5\right\} \subseteq [1, \infty)$. Let us call $\alpha \in \mathbb{R}$ an $X$-sum if there exists $A \subseteq X$ such that $\sum_{a \in A} a= \alpha$. Prove that if $\alpha_1, \ldots, \alpha_{11}$ are 11 distinct $X$-sums, then there exists $i, j$ such that $\left|a_i-a_j\right| \geqslant 1$. 
\end{enumerate}


\section*{Question 4}

\begin{enumerate}
    \item[(a)] (\textbf{10 marks}) Let $s, t \geqslant 2$ be integers. Define the Ramsey number $R(s, t)$ and prove that it satisfies
    \[
    R(s, t) \leqslant 
    {s+t-2 \choose
    s-1}
    .
    \]
    \item[(b)] (\textbf{5 marks}) Let $R_k(3)$ be the smallest integer $n$ such that any colouring of the edges of $K_n$ with $k$ colours contains a monochromatic $K_3$. Using the fact that $R(3,3)=6$ or otherwise show that $R_3(3) \leqslant 17$.
    \item[(c)] (\textbf{10 marks}) Prove that if $n \geqslant s$ satisfy
    \[
    {n \choose s} < 4^{{s \choose 2} - 1},
    \]
    then $R_4(s)>n$.
\end{enumerate}







\end{document}
